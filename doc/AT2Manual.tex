\documentclass[acus]{article}



\usepackage{booktabs} 
\usepackage{longtable}
\usepackage{subfig}

\usepackage{graphicx}

\begin{document}

\title{AT 2.0 Manual}
\maketitle
\begin{abstract}
We give an overview for the AT 2.0 code and describe how the functions and repository fit into this.
\end{abstract}


\section{Introduction}

\section{Lattice Creation}
The element creation functions are the following:
\begin{itemize}
\item atdrift \ \ Class: Drift
\item atmonitor \ \ Class: Monitor
\item atmultipole \ \ Class: Multipole
\item atthinmultipole \ \ Class: 
\item atquadrupole  \ \ Class: Quadrupole
\item atrbend  \ \ Class: Bend 
\item atrfcavity \ \ Class: RFCavity
\item atsbend \ \ Class: Bend
\item at solenoid \ \ Class: Solenoid
\item atsextupole  \ \ Class: Sextupole
\item atwiggler  \ \ Class: Wiggler
\item idtable  \ \ Class:  KickMap
\end{itemize}


\section{Pass Methods}

\begin{itemize}
\item BndMPoleSymplectic4E2Pass
\item BndMPoleSymplectic4E2RadPass
\item BndMPoleSymplectic4Pass
\item BndMPoleSymplectic4RadPass
\item BendLinearPass
\item CavityPass
\item CorrectorPass
\item DriftPass
\item QuadLinearPass
\item QuadMPoleFringePass
\item StrMPoleSymplectic4Pass
\item StrMPoleSymplectic4RadPass
\item ThinMPolePass
\item WiggLinearPass
\item IDTablePass
\end{itemize}



\section{Lattice Manipulation}
A lattice manipulation function takes a lattice as an argument and produces a new lattice as a result.
Here are some lattice manipulation functions:
\begin{itemize}
\item atsetshift
\item atsettilt
\item atsetfieldvalues
\item ataddmpolecomppoly
\item ataddmpoleerrors
\item atloadfielderrs
\end{itemize}


\section{Tracking Particles plus Moments}
The pass methods have two different calling methods.  They may be called directly via the Mex interface (through the MexFunction entry point in the C function), or they may be called indirectly through the function RingPass (through the passFunction entry).  The pass methods should be defined so that calling them with no arguments gives a list of required and optional parameters.

The moment tracking and equilibrium finding occurs via the function OhmiEnvelope().  For this to work requires pass methods that include radiation (this gives a deterministic effect which results in damping and non-symplecticity.  Further, the function findmpoleraddiffmatrix is required to compute the diffusion matrix.

\section{Lattice Functions}
Given the ability to track particles through the lattice, one can compute beam dynamics properties around the ring.  The closed orbit, or fixed point of stable motion is one example.  Next there are the various ways to characterize the betatron and synchrotron oscillations.  Twiss parameters plus various coupled generalizations may be computed.

\section{Visualization}
The lattice functions described in the previous section may be plotted, together with a synoptic representation of the lattice.  The function atplot is designed for this purpose.

\section{AT within a larger context: Other Codes, Matlab Middle Layer}

\begin{thebibliography}{2}

\bibitem{KMW}
A. Terebilo \emph{Accelerator Toolbox for Matlab}, SLAC-PUB 8732 (May 2001)

\end{thebibliography}

\end{document}

